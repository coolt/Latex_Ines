%%%%%%%%%%%%%%%%%%%%%%%%%%%%%%%%%%%%%%%%%%%%%%%%%%%%%%%%%%%%%%%%%
%  _____       ______   ____																		%
% |_   _|     |  ____|/ ____| 		Institute of Embedded Systems												%
%   | |  _ __ | |__  | (___    		Wireless Group														%
%   | | | '_ \|  __|  \___ \   		          Zuercher Hochschule Winterthur												%
%  _| |_| | | | |____ ____) |  		(University of Applied Sciences)												%
% |_____|_| |_|______|_____/  		8401 Winterthur, Switzerland												%
%																						%
%%%%%%%%%%%%%%%%%%%%%%%%%%%%%%%%%%%%%%%%%%%%%%%%%%%%%%%%%%%%%%%%%


% The ZHAW INES latex style scrip
\documentclass{clsFile/zhawines}


%INclude your own packages that are needed for you document, here the packages for the example are added
\usepackage{scrpage2}
\usepackage{listings}
\usepackage{amsmath}
\usepackage[squaren]{SIunits}
\usepackage{graphicx}
\usepackage{array}
\usepackage{float}

\begin{document}

%How the hyperlinks are shown, remove this or change it if you want the colorful links
\hypersetup{
    colorlinks = false,
    urlbordercolor = {1 1 1},
    linkbordercolor = {white},
}

%Select language, all generated content will be in the selected language
\selectlanguage{german} %possible german/english

% Generate your title page, it takes the arguments author, title
% and email. The title page is then generated with maketitle
\author{Autor1 \newline Autor2}
\title{Latex Vorlage}
\email{author@zhaw.ch}
\newcommand{\version}{0.1}

% Add Copyright informations and Contact infos
\newcommand{\confidential}{yes}  %write yes to have a confidential mark on every normal page

\maketitle


% Generate the header and footer of the document
\headings

%Include table of content
\tableofcontents

% Starts the page count from here, pdf viewers will be able to jump to the corret page.
 \setcounter{page}{1}

%Include all chapters. Chapters itself are in the content subfolder
\include{content/howtoLaTeX} % Für das Schlussdokument auskommentieren
%%%%%%%%%%%%%%%%%%%%%%%%%%%%%%%%%%%%%%%%%%%%%%%%%%%%%%%%%%%%%%%%%
%  _____       ______   ____									%
% |_   _|     |  ____|/ ____|  Institute of Embedded Systems	%
%   | |  _ __ | |__  | (___    Wireless Group					%
%   | | | '_ \|  __|  \___ \   Zuercher Hochschule Winterthur	%
%  _| |_| | | | |____ ____) |  (University of Applied Sciences)	%
% |_____|_| |_|______|_____/   8401 Winterthur, Switzerland		%
%																%
%%%%%%%%%%%%%%%%%%%%%%%%%%%%%%%%%%%%%%%%%%%%%%%%%%%%%%%%%%%%%%%%%

\chapter{Einleitung}\label{chap.einleitung}


\section{Ausgangslage}\label{ausgangslage}

\begin{itemize}
\item Nennt bestehende Arbeiten/Literatur zum Thema -> Literaturrecherche
\item Stand der Technik: Bisherige Lösungen des Problems und deren Grenzen
\item (Nennt kurz den Industriepartner und/oder weitere Kooperationspartner und dessen/deren Interesse am Thema Fragestellung)
\end{itemize}



\section{Zielsetzung / Aufgabenstellung / Anforderungen}\label{zielsetzung}

\begin{itemize}
\item Formuliert das Ziel der Arbeit
\item Verweist auf die offizielle Aufgabenstellung des/der Dozierenden im Anhang
\item (Pflichtenheft, Spezifikation)
\item (Spezifiziert die Anforderungen an das Resultat der Arbeit)
\item (Übersicht über die Arbeit: stellt die folgenden Teile der Arbeit kurz vor)
\item (Angaben zum Zielpublikum: nennt das für die Arbeit vorausgesetzte Wissen)
\item (Terminologie: Definiert die in der Arbeit verwendeten Begriffe)
\end{itemize}

\include{content/grundlagen} % (2. Theoretische Grundlagen)
%%%%%%%%%%%%%%%%%%%%%%%%%%%%%%%%%%%%%%%%%%%%%%%%%%%%%%%%%%%%%%%%%
%  _____       ______   ____									%
% |_   _|     |  ____|/ ____|  Institute of Embedded Systems	%
%   | |  _ __ | |__  | (___    Wireless Group					%
%   | | | '_ \|  __|  \___ \   Zuercher Hochschule Winterthur	%
%  _| |_| | | | |____ ____) |  (University of Applied Sciences)	%
% |_____|_| |_|______|_____/   8401 Winterthur, Switzerland		%
%																%
%%%%%%%%%%%%%%%%%%%%%%%%%%%%%%%%%%%%%%%%%%%%%%%%%%%%%%%%%%%%%%%%%

\chapter{Vorgehen / Methoden}\label{chap.vorgehen}


\begin{itemize}
\item (Beschreibt die Grundüberlegungen der realisierten Lösung (Konstruktion/Entwurf) und die Realisierung als Simulation, als Prototyp oder als Software-Komponente)
\item (Definiert Messgrössen, beschreibt Mess- oder Versuchsaufbau, beschreibt und dokumentiert Durchführung der Messungen/Versuche)
\item (Experimente)
\item (Lösungsweg)
\item (Modell)
\item (Tests und Validierung)
\item (Theoretische Herleitung der Lösung)
\end{itemize}



\section{(Verwendete Software)}\label{software}
Für die vorliegende Arbeit wurden die unten aufgeführten Programme eingesetzt.

\subsection*{Arbeitsumgebung}\label{wintool}
\begin{itemize}
	\item Microsoft Windows 8 developer preview
\end{itemize}

\subsection*{Virtual Machine}\label{vm}
\begin{itemize}
	\item Oracle VM VirtualBox, Version 3.2.10
\end{itemize}

\subsection*{CAD Catia}\label{catia}
\begin{itemize}
	\item CATIA, Version 5.19 (in VirtualBox)
\end{itemize}

\subsection*{Dokumentation}\label{dokutools}
\begin{itemize}
	\item proTeXt mit TexMakerX 2.1 (SVN 1774), \href{http://www.latex-project.org/ftp.html}{latex-project.org}
	\item Microsoft Visio 2007
	\item Adobe Acrobat 8 Professional 8.1.6
\end{itemize}
 % Vorgehen / Methoden
%%%%%%%%%%%%%%%%%%%%%%%%%%%%%%%%%%%%%%%%%%%%%%%%%%%%%%%%%%%%%%%%%
%  _____       ______   ____									%
% |_   _|     |  ____|/ ____|  Institute of Embedded Systems	%
%   | |  _ __ | |__  | (___    Wireless Group					%
%   | | | '_ \|  __|  \___ \   Zuercher Hochschule Winterthur	%
%  _| |_| | | | |____ ____) |  (University of Applied Sciences)	%
% |_____|_| |_|______|_____/   8401 Winterthur, Switzerland		%
%																%
%%%%%%%%%%%%%%%%%%%%%%%%%%%%%%%%%%%%%%%%%%%%%%%%%%%%%%%%%%%%%%%%%

\chapter{Resultate}\label{chap.resultate} 

\begin{itemize}
\item (Zusammenfassung der Resultate)
\end{itemize}

%%%%%%%%%%%%%%%%%%%%%%%%%%%%%%%%%%%%%%%%%%%%%%%%%%%%%%%%%%%%%%%%%
%  _____       ______   ____									%
% |_   _|     |  ____|/ ____|  Institute of Embedded Systems	%
%   | |  _ __ | |__  | (___    Wireless Group					%
%   | | | '_ \|  __|  \___ \   Zuercher Hochschule Winterthur	%
%  _| |_| | | | |____ ____) |  (University of Applied Sciences)	%
% |_____|_| |_|______|_____/   8401 Winterthur, Switzerland		%
%																%
%%%%%%%%%%%%%%%%%%%%%%%%%%%%%%%%%%%%%%%%%%%%%%%%%%%%%%%%%%%%%%%%%

\chapter{Diskussion und Ausblick}\label{chap.diskussion}

\begin{itemize}
\item Bespricht die erzielten Ergebnisse bezüglich ihrer Erwartbarkeit, Aussagekraft und Relevanz
\item Interpretation und Validierung der Resultate
\item Rückblick auf Aufgabenstellung, erreicht bzw. nicht erreicht
\item Legt dar, wie an die Resultate (konkret vom Industriepartner oder weiteren Forschungsarbeiten; allgemein) angeschlossen werden kann; legt dar, welche Chancen die Resultate bieten
\end{itemize}
 % Diskussion und Ausblick


% Generate the bibliography and add it to the table of contents, there is an extra bib file where you can add entries(subfolder BibTeX-Literaturverzeichnis)
\addcontentsline{toc}{chapter}{\bibname}
\bibliographystyle{unsrt}
\bibliography{BibTex/literatur}

% Additional lists
\listoffigures
\listoftables
%%%%%%%%%%%%%%%%%%%%%%%%%%%%%%%%%%%%%%%%%%%%%%%%%%%%%%%%%%%%%%%%%
%  _____       ______   ____									%
% |_   _|     |  ____|/ ____|  Institute of Embedded Systems	%
%   | |  _ __ | |__  | (___    Wireless Group					%
%   | | | '_ \|  __|  \___ \   Zuercher Hochschule Winterthur	%
%  _| |_| | | | |____ ____) |  (University of Applied Sciences)	%
% |_____|_| |_|______|_____/   8401 Winterthur, Switzerland		%
%																%
%%%%%%%%%%%%%%%%%%%%%%%%%%%%%%%%%%%%%%%%%%%%%%%%%%%%%%%%%%%%%%%%%

\chapter*{Abkürzungsverzeichnis}\label{chap.glossar}
\addcontentsline{toc}{section}{(Abkürzungsverzeichnis)}\label{cha:abkürzungsverzeichnis}

In diesem Abschnitt werden Abkürzungen und Begriffe kurz erklärt.

\begin{longtable}{|m{3cm}|m{11cm}|}\hline	
	\rowcolor{gray} \textbf{Abk}&
	Abkürzung \\ \hline	

	\textbf{XY}&
	Ix Ypsilon \\ \hline	
	
	\textbf{YZ}&
	Ypsilon Zet \\ \hline
	
	
	
	
	
%\caption{Abkürzungs Verzeichnis}
%\label{tab:glossar}
\end{longtable}

\lstlistoflistings

\include{content/anhang} % Anhang

\end{document}
